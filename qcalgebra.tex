\documentclass[a4paper,11pt]{amsbook}
\usepackage{amsmath}
\usepackage{amsfonts}
\usepackage{amssymb}
\begin{document}
\title{QCAlgebra Package}
\author{Fausto Saporito}
\maketitle
\tableofcontents
\part{Description}

\chapter{Package description}

\section{Generalities}

These verbs (written in J) implements some quantum computing algebra useful to 
perform some quantum circuit calculation.\\

In this schema 
\begin{itemize}
\item K0 is $|0\rangle$ 
\item K1 is $|1\rangle$ 
\end{itemize}

and K00($|00>$) is K0 TP K0. The verb TP is used to create bigger qubits (Tensor Product).\\

Internally, the qubits are expressed via boxed set (the first element is the complex coefficient). So, for example,

\begin{itemize}
\item 1;0 0 means $|00\rangle $
\item 1;0 1 0 means $|010\rangle $
\item 2;0 0 means $2|00\rangle $
\end{itemize}

\section{Gates}

The GATES are written in CAPITAL LETTERS (they performs automatically a simplification and cleaning) and the arguments follow the J-standard:
\begin{enumerate}
\item HD : Hadamard Gate acting on qubit x of quantum state y
\begin{center}
 0 HD K00
\end{center}
it means Hadamard gate acting on the first qubit (qubit 0) of $|00\rangle$, so $\frac{1}{\sqrt{2}}(|0\rangle  + |1\rangle )|0\rangle  = \frac{1}{\sqrt{2}}(|00\rangle  + |10\rangle )$. This gate is useful to generate a superposition of states, starting from a simple one ($|0\rangle$ or $|1\rangle$).
\item XG, YG, ZG: Pauli Gates. They use the same standard as described for HD gate
\item CNOT : Controlled NOT (XG) gate. Use: (c,t) CNOT qreg, where c is control qubit and t is target qubit
\item CY, CZ : Controlled Y and Controlled Z. As above.
\item R(phi) : Phase shift gate
\item Rk : Phase shift gate modified for QFT (internal use)
\item CRK : Controlled version of Rk
\end{enumerate}

There are some other interesting verbs, used mainly inside the other verbs to 
perform very specific activities. They are:
\begin{itemize}
\item simpl : it performs simplification of qubit expression (summation and cleaning of small complex amplitudes)
\item PROB : it extracts the probability related to measurement of bits in the qreg, i.e. (0;1 1) 0 HD K00 - it returns the probability to have as result of measurement the first two qubit to 1, so all the states: $|11?\rangle$ after applying hadamard gate to the first qubit of $|00\rangle$. The first element in the boxed list is the offset (0-based) for the qubit pattern.
\item QFT : performs the QFT (quantum fourier transform) on qubits. 3 QFT K000 computes the QFT on the three first qubits of K000, in terms of qubits we have:\\
\[
\begin{array}{}
$0.35355339+0.00000000i |000\rangle + $ \cr
$0.35355339+0.00000000i |001\rangle + $ \cr
$0.35355339+0.00000000i |010\rangle + $ \cr
$0.35355339+0.00000000i |011\rangle + $ \cr
$0.35355339+0.00000000i |100\rangle + $ \cr
$0.35355339+0.00000000i |101\rangle + $ \cr
$0.35355339+0.00000000i |110\rangle + $ \cr
$0.35355339+0.00000000i |111\rangle$
\end{array}
\]
\end{itemize}

Last but not least, there's a starting implementation of QEC protocols.
At the moment there's an encoding developed by Shor, using 9 qubits.


\end{document}